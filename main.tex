\documentclass[12pt]{article}
\usepackage[a4paper, margin=1in]{geometry}
\usepackage{hyperref}
\usepackage{graphicx}

\begin{document}

\title{React Native TodoList App}
\author{Deepak Naga Sai Vivek Kancharla}
\date{November 17th 2024}
\maketitle

\section*{1. Screenshots of Your App}
\subsection*{Emulator Screenshot}
Take a screenshot of your app running on an emulator (e.g., Android Studio Emulator or iOS Simulator).

\includegraphics[width=\textwidth]{Image1.jpeg}


\subsection*{Physical Device Screenshot}
\includegraphics[width=\textwidth]{Image06.png}
\subsection*{Differences Observed}
\begin{itemize}
    \item \textbf{Performance:} The app typically runs faster on physical devices due to direct hardware usage, while emulators may experience lag.
    \item \textbf{Device-Specific Features:} Physical devices allow testing real-world features like camera, GPS, or touch gestures, which might not behave the same in an emulator.
    \item \textbf{Screen Resolution:} Physical devices display apps at native resolution, which may differ from emulator settings.
\end{itemize}

\section*{2. Setting Up an Emulator}
\subsection*{Steps Followed}
\textbf{For iOS:}
\begin{enumerate}
    \item Install Xcode from the App Store.
    \item Open Xcode > Preferences > Platforms and download the necessary simulators.
    \item Launch the iOS Simulator from Xcode or the terminal with:
    \begin{verbatim}
    xcrun simctl list
    xcrun simctl boot <device-id>
    \end{verbatim}
\end{enumerate}

\subsection*{Challenges and Solutions}
\begin{itemize}
    \item \textbf{Challenge:} Emulator stuck at boot screen.\\
    \textbf{Solution:} Increase RAM allocation in the emulator settings or use a different system image.
    \item \textbf{Challenge:} Slow emulator performance.\\
    \textbf{Solution:} Enable hardware acceleration (HAXM for Intel processors or Hypervisor Framework for macOS).
\end{itemize}

\section*{3. Running the App on a Physical Device Using Expo}
\subsection*{Steps Followed}
\begin{enumerate}
    \item Install the Expo CLI:
    \begin{verbatim}
    npm install -g expo-cli
    \end{verbatim}
    \includegraphics[width=\textwidth]{image04.png}
    \item Create a project:
    \begin{verbatim}
    npx expo init YourProjectName
    \end{verbatim}
    \item Start the Expo development server:
    \begin{verbatim}
    npx expo start
    
    \end{verbatim}
    \includegraphics[width=\textwidth]{Screenshot 1.png}
    \item Install the Expo Go app on your device (from App Store or Google Play).
    \item Scan the QR code displayed in the Expo CLI using the Expo Go app.
    \item Your app runs on the device immediately after scanning.
\end{enumerate}

 
\subsection*{Troubleshooting Steps}
\begin{itemize}
    \item \textbf{Issue:} QR code not working.\\
    \textbf{Solution:} Ensure both the device and computer are on the same Wi-Fi network.
    \item \textbf{Issue:} App crashes on the device.\\
    \textbf{Solution:} Check for incompatible dependencies and reinstall node modules.
\end{itemize}

\section*{4. Comparison of Emulator vs. Physical Device}
\subsection*{Emulator}
\textbf{Advantages:}
\begin{itemize}
    \item Easy to set up and use.
    \item Debugging tools are integrated (e.g., logcat in Android Studio).
    \item No need for physical hardware.
\end{itemize}

\textbf{Disadvantages:}
\begin{itemize}
    \item Slower performance, especially with animations or heavy computations.
    \item Limited support for device-specific features like Bluetooth or real-world GPS.
\end{itemize}

\subsection*{Physical Device}
\textbf{Advantages:}
\begin{itemize}
    \item Real-world performance and feature testing.
    \item Faster and smoother UI/UX experience.
    \item Accurate testing of hardware features (e.g., camera, accelerometer).
\end{itemize}

\textbf{Disadvantages:}
\begin{itemize}
    \item Requires USB debugging or Expo setup.
    \item Dependency on device model for testing compatibility.
\end{itemize}

\section*{5. Troubleshooting a Common Error}
\subsection*{Error Encountered}
\textbf{Issue:} \texttt{EADDRINUSE: address already in use :::8081}.

\subsection*{Cause}
The default Metro bundler port \texttt{8081} was already being used by another process.

\subsection*{Steps Taken to Resolve}
\begin{enumerate}
    \item Identify the process using the port:
    \begin{verbatim}
    lsof -i :8081
    \end{verbatim}
    \item Kill the conflicting process:
    \begin{verbatim}
    kill -9 <PID>
    \end{verbatim}
    \item Alternatively, use a different port for Metro:
    \begin{verbatim}
    npx react-native start --port=8088
    \end{verbatim}
\end{enumerate}




\title{Task 2 Submission: Building a Simple To-Do List App}
\author{}
\date{}
\maketitle


\section{Mark Tasks as Complete}

\subsection{Implementation Details}

\textbf{Toggle Functionality}:
\begin{itemize}
    \item Added a \texttt{toggleComplete} function that updates the completion status of tasks in the state.
    \item Tasks have a \texttt{completed} property in their state object, initialized as \texttt{false}.
    \item When a task is toggled, the state is updated using the \texttt{setTasks} function.
\end{itemize}

\textbf{Styling Completed Tasks}:
\begin{itemize}
    \item Used conditional styling to apply a strikethrough text style or change the text color for completed tasks.
    \item Example:
    \begin{verbatim}
const taskStyle = {
  textDecorationLine: task.completed ? 'line-through' : 'none',
  color: task.completed ? 'gray' : 'black'
};
    \end{verbatim}
\end{itemize}

\subsection{State Update Code}
\begin{verbatim}
const toggleComplete = (id) => {
  setTasks(tasks.map(task =>
    task.id === id ? { ...task, completed: !task.completed } : task
  ));
};
\end{verbatim}



\section{Persist Data Using AsyncStorage }

\subsection{Implementation Details}

\textbf{Saving Data}:
\begin{itemize}
    \item Used \texttt{AsyncStorage} to store the tasks list when the state changes.
    \item Added a \texttt{useEffect} to save data on every update.
\end{itemize}

\textbf{Retrieving Data}:
\begin{itemize}
    \item On app launch, retrieved the stored tasks from \texttt{AsyncStorage} and initialized the state.
\end{itemize}


\begin{verbatim}
import AsyncStorage from '@react-native-async-storage/async-storage';

useEffect(() => {
  const loadTasks = async () => {
    const storedTasks = await AsyncStorage.getItem('tasks');
    if (storedTasks) setTasks(JSON.parse(storedTasks));
  };
  loadTasks();
}, []);

useEffect(() => {
  AsyncStorage.setItem('tasks', JSON.stringify(tasks));
}, [tasks]);
\end{verbatim}

\includegraphics[width=\textwidth]{Image2.jpeg}
\section{Edit Tasks }

\subsection{Implementation Details}

\textbf{Editing Functionality}:
\begin{itemize}
    \item Added functionality to tap a task to enter edit mode.
    \item In edit mode, the task’s text is replaced with an editable \texttt{TextInput} component.
\end{itemize}

\textbf{Updating State}:
\begin{itemize}
    \item Implemented an \texttt{updateTask} function to modify the content of a task in the state array.
    \item Used a temporary state variable to track the current editing task.
\end{itemize}


\begin{verbatim}
const [editingTask, setEditingTask] = useState(null);
const [newText, setNewText] = useState('');

const updateTask = (id) => {
  setTasks(tasks.map(task =>
    task.id === id ? { ...task, text: newText } : task
  ));
  setEditingTask(null);
};
\end{verbatim}

\subsection{UI Management}
\begin{itemize}
    \item Rendered a \texttt{TextInput} when \texttt{editingTask} equals the task ID.
    \item Rendered regular text otherwise.
\end{itemize}


\section{Add Animations }

\subsection{Implementation Details}

\textbf{Adding Animations}:
\begin{itemize}
    \item Used the React Native \texttt{Animated} API to enhance user experience.
    \item Added animations for task addition and deletion using \texttt{Animated.Value}.
\end{itemize}

\textbf{Task Addition}:
\begin{itemize}
    \item Applied a fade-in effect using \texttt{Animated.timing}.
    \item Code:
    \begin{verbatim}
const fadeIn = new Animated.Value(0);

const animateTask = () => {
  Animated.timing(fadeIn, {
    toValue: 1,
    duration: 500,
    useNativeDriver: true
  }).start();
};
    \end{verbatim}
\end{itemize}

\textbf{Task Deletion}:
\begin{itemize}
    \item Implemented a slide-out effect before removing the task from the state.
\end{itemize}
\includegraphics[width=\textwidth]{Image3.jpeg}



\subsection{Description}
\begin{itemize}
    \item \textbf{Fade-in Animation}: Tasks gradually appear when added.
    \item \textbf{Slide-out Animation}: Tasks slide out to the left when deleted.
    \item These animations make the app feel more dynamic and engaging.
\end{itemize}

\section{Conclusion}
The React Native To-Do List app was extended with the following features:
\begin{enumerate}
    \item \textbf{Mark Tasks as Complete}: Users can toggle the completion status of tasks with updated styles.
    \item \textbf{Persist Data Using AsyncStorage}: Task data is stored and retrieved seamlessly, ensuring persistence.
    \item \textbf{Edit Tasks}: Tasks can be edited directly within the app, enhancing flexibility.
    \item \textbf{Add Animations}: Visual effects make the app more engaging and professional.
\end{enumerate}

These enhancements greatly improved the app's usability and user experience.


\subsection*{ GitHub Repository}
\begin{itemize}
    \item https://github.com/SaiVivekKancharla/TodoListApp
\end{itemize}

\end{document}



